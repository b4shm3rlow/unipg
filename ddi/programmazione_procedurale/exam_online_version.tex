\documentclass[a4paper,12pt] {exam}
\usepackage{listings}
\lstset{language=C,numbers=left, numberstyle=\tiny, stepnumber=1, numbersep=5pt}

\begin{document}
\begin{center}
\fbox{\fbox{\parbox{5.5in}{\centering
Esempio Prova scritta Programmazione I - Online Version}}}
\end{center}
\vspace{0.1in}
\hbox to \textwidth{Nome e Cognome:\enspace\hrulefill}
\hbox to \textwidth{Matricola:\enspace\hrulefill}
\begin{questions}

\question \fbox{1 punto} Dato il seguente codice, trovare ,se ci sono, possibili errori nelle dichiarazioni delle variabili.
\vspace{0.1in}
\begin{lstlisting}
int main (void) {

float p* = NULL;
int a = 1;
char @mail = 'M';
double A = 1,5;
int b = 0;
long int b = 0;
}
\end{lstlisting}


\question \fbox{1 punto} Definite successivamente le variabili \textit{x} e \textit{y}, la stampa di \textit{ERROR!} viene eseguita correttamente?
\vspace{0.1in}
\begin{lstlisting}
unsigned short x = 5;
int y = -1;
if (x > y)
	printf("ERROR!\n");
\end{lstlisting}

\question \fbox{1 punti} Data la seguente porzione di codice, specificare le conversioni di tipo.
\begin{lstlisting}
float f = 10L + 0x1f;
char s = '$';
unsigned int u = 1.5 + s;
int d = f * s;
	
\end{lstlisting}

\question \fbox{1 punto} Dato il seguente codice, quale sarà il valore della stampa della printf?
\begin{lstlisting}
	int main(){
		int a = 0;
		double d = 10.21;
		printf("%lu", sizeof(a + d));
		return 0;
	}
	
\end{lstlisting}
\pagebreak
\question \fbox{2 punti} Quale sarà l'output del seguente codice C ?
\begin{lstlisting}
#include<stdio.h>
#define MAX(a,b,c) (a>b ? a>c ? a: c: b>c ? b : c)
	
int main() {
	int x;
	x = MAX(3+2, 2+7, 4+6);
	printf("%d", x);
	return 0;
}
\end{lstlisting}

\question \fbox{2 punti} Data la breve porzione di codice, la stampa viene eseguita correttamente? Perchè?
\begin{lstlisting}
	#include<stdio.h>
	main()
	{
		printf("%d",strcmp("strcmp()","strcmp()"));
	}
\end{lstlisting}

\question \fbox{2 punti} Dire quali tra i seguenti assegnamenti sono legali e quali sono illegali (cioè che portano a warning/errori).
\begin{lstlisting}
int *i;
float *f;
void *v;
i = ( double *) 4;
i = v = f;
i = f;
v = 1;
i = 5;
f = ( float *) f;
\end{lstlisting}

\question \fbox{2 punti} Dato il seguente codice, cosa rappresenta "pippo"?
\begin{lstlisting}
	#include<stdio.h>
	
	int main () {
		
		typedef char (*(*aaa[3])())[10];
		aaa pippo
		return 0;
	}
	
\end{lstlisting}
\pagebreak[5]
\question \fbox{2 punti} Descrivere brevemente cosa esegue il seguente programma
\begin{lstlisting}
	int main() {		
		
		fork();
		fork();
		printf("Hello world\n");
	}
\end{lstlisting}

\question \fbox{2 punti} Considerando la porzione di programma, cosa verrà stampato dalla \textit{printf}?

\begin{lstlisting}
	int v[] = {30, 25, 12, 15, 4}, x = v[0], i;
	for (i = 1; i < 4; i++)
	if(v[i] < x) x = v[i];
	printf("%d", x);
	return 0;
\end{lstlisting}

\question \fbox{2 punti} Dato il seguento programma, quante volte verrà stampata la scritta "Hello"?
\begin{lstlisting}
	int main () {
		
		int x;		
		for(x=-1; x<=20; x++)int i;
		{
			if(x < 10)
			continue;
			else
			break;
			printf("Hello");
		}
	}
\end{lstlisting}
\pagebreak
\question \fbox{3 punti} Qual'è l'output del programma?
\begin{lstlisting}
	#include<stdio.h>
	int main()
	{
		const int x=5;
		const int *ptrx;
		ptrx = &x;
		*ptrx = 10;
		printf("%d\n", x);
		return 0;
	}
\end{lstlisting}
\pagebreak[5]
\question \fbox{3 punti} Data la seguente porzione di codice, quante volte viene stampato \textit{"Hello World!"}? Quale saranno il valore di \textit{a} e \textit{b} nell'ultima stampa?
\begin{lstlisting}
	int main()
	{
		int a = 0x11 - 3;
		int c=0;
		while ((a-=1)? a++:a--) {
			c+=1;
			a-=1;
			printf("Hello World! %d\n",c);
		}
		int b = 0xaa;
		b+= a || a --;
		printf("%d %d\n", a, b);
	}
\end{lstlisting}

\question \fbox{3 punti} Data il seguente programma, segnalare l'errore nel codice.
\begin{lstlisting}
	int main() {
		
		struct emp {
		char name[25]; int age;};
		struct emp e;
		e.name = "Pippo";
		e.age = 18;
		printf("%s %d", e.name, e.age);
		return 0;
	}
\end{lstlisting}
\question \fbox{3 punti} Il seguente programma esegue la stampa correttamente o da errore di compilazione? Se stampa, quale sarà il suo output?
\begin{lstlisting}
	#include<stdio.h>
	
	main() { 
		struct { int x;} var = {5}, *p = &var;
		
		printf("%d %d %d",var.x,p->x,(*p).x); 
	}
	
\end{lstlisting}
\end{questions}

\end{document}